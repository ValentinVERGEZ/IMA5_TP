\section{Dimensionnement sans optimisation}
\subsection{Cahier des charges}
Nous allons réaliser le dimensionnement d'un transformateur monophasé, c'est à dire déterminer la section de son noyau, la section de la fenêtre des bobinages, le nombre de spires et la section du fil de cuivre pour les enroulements primaires et secondaires.\\
On retrouve en Figure \ref{img:cotesTransfo} le dessin du transformateur avec ses côtes.\\

\begin{figure}[h]
	\begin{center}
	\includegraphics[width=0.5\textwidth]{images/dessin_init_transfo.jpg}
	\caption{Côtes du transformateur}\label{img:cotesTransfo}
	\end{center}
\end{figure}
\FloatBarrier

Nous allons suivre le cahier des charges suivant : \\~\\
\begin{tabular}{l l}
 $\bullet $ $V_1 = 230 V$	&(Tension primaire nominale)\\
 $\bullet $ $V_2 = 127 V$	&(Tension secondaire nominale)\\
 $\bullet $ $S = 1500 VA$	&(Puissance nominale nominale)\\
 $\bullet $ $J_{max} = 5 A/mm^2$	&(Densité de courant maximale)\\
 $\bullet $ $f = 50Hz$		&(Fréquence de fonctionnement)\\
 $\bullet $ $K_v = \cfrac{2\Pi}{\sqrt{2}}$	&(Facteur de forme, onde sinusoïdale)\\
 $\bullet $ $K_u = 0,8 $	&(Coefficient de remplissage de la fenêtre de bobinage)\\
 $\bullet $ $B_{max} = 1T$	&(Induction maximale)
\end{tabular}

	

\subsection{Équation de dimensionnement}

Démonstration de la formule suivante : $ Sfer = \cfrac{2S}{K_u.K-v.f.B_{max}.J_{max}.W_a}$


Pour étudier le phénomène, appliquons la loi de Faraday qui donne la f.é.m. aux bornes du circuit : 

$e = n= \cfrac{d\varphi(t)}{dt}$

Si on néglige la résistance du circuit, ainsi que les fuites de flux, la source v compensera à chaque instant cette f.é.m. : \\
$s = e = V_{max}\sin(\omega{}t) = n\cfrac{d\varphi(t)}{dt}$\\


On peut calculer le flux créé par le circuit à l’intérieur du noyau par intégration : \\
$\varphi(t) = \int {\cfrac{V_{max}}{n} \sin(\omega{}t) dt} = \cfrac{V_{max}}{n\omega}\cos((\omega{}t)) + \varphi_0$

La constante d'intégration $\varphi_0$ peut correspondre à la présence d'un aimant permanent, à l’existence d'un flux rémanent ou encore à la présence d'un autre enroulement alimenté en courant continu. 

Dans notre cas nous considérons que $\phi_0 = 0$\\

$\varphi(t) = - \cfrac{V_{max}}{n\omega} \cos(\omega{}t) = \cfrac{V_{max}}{n\omega} \sin(\omega{}t - \cfrac{\Pi}{2})$\\


Le déphasage de $- \cfrac{\Pi}{2}$ indique que le flux est en quadrature retard sur la tension $v =  V_{max}\sin(\omega{}t)$\\

D'après la relation $n\varphi = Li$, on en déduit que le courant passant dans la bobine s'écrit : \\

$i = \cfrac{V_{max}}{L\omega} \sin(\omega{}t - \cfrac{\Pi}{2})$, on retrouve ka définition de l'impédance d'une bobine parfait :\\

$Z = L\omega$


Si la section est constante, le champ B dans le noyau vaut : \\

$B(t) =  \cfrac{\varphi(t)}{S} = \cfrac{V_{max}}{n\omega S}\sin(\omega{}t - \cfrac{\Pi}{2})$\\


On voit que la valeur maximale de ce champ est reliée à la valeur efficace de la f.é.m. par la relation :\\

$B_{max} =  \cfrac{V_{max}}{n\omega S} = \cfrac{sqrt{2}V_{eff}}{n 2\Pi f S}$ \\

que l'on peut écrire : \\

$V_{eff} = 4.44 n f S B_{max} = 4.44 n f \phi_{max}$\\


On appelle cette relation formule de Boucherot. Cette expression montre que la valeur maximale du flux $\phi_{max}$ ne dépend que de la valeur efficace $V_{eff} $ de la tension d'alimentation (à fréquence constante) La tension d'alimentation impose le flux et l'enroulement appelle un courant en conséquence. 

	
\subsection{Hypothèses et résultats}
Nous avons une équation pour trois inconnus : b, H ($b.H = S_{fer}$) et hf ($\frac{b}{2}.hf = W_a$).\\
Premièrement nous allons fixer définitivement une forme de noyau ferromagnétique. Le noyau sera carré et ainsi $b = H$.\\
Puis, pour cette première partie, nous allons fixer la section de fenêtre des bobinages, $W_a = 2\times{}10^{-3} m²$.
Maintenant le problème ne présente plus qu'une unique solution.\\

\newpage
Après la résolution via un algorithme Matlab, nous obtenons les résultats suivants : \\
\begin{itemize}
\item $S_{fer} = 0,0017 m² $
\item $b = 0,0411 m $
\item $H = 0,0411 m $
\item $hf = 0,0974 m $
\end{itemize}~\\

Cela donne le dessin de transformateur suivant :
\begin{figure}[h]
	\begin{center}
	\includegraphics[width=0.4\textwidth]{images/TP1_transfo_carre}
	\caption{Dessin du transformateur dimensionné sans optimisation}\label{img:dessinTransfoCarre}
	\end{center}
\end{figure}
\FloatBarrier 

Dans ce cas de figure nous avons les volumes et masses suivantes : \\
\begin{itemize}
\item $V_{fer} = 5,368 \times{}10^{-4} m^3$
\item $V_{cu} = 3,662 \times{}10^{-4} m^3$
\item $M_{fer} = 4,22 kg$
\item $M_{cuivre} = 3,27 kg $
\item $M_{totale} = 7,49 kg $
\end{itemize}~\\

\newpage
\section{Dimensionnement avec optimisation du poids}
\subsection{Nouvelles côtes}
Nous avons fait un premier dimensionnement en fixant arbitrairement la section  de la fenêtre des bobinages. Maintenant nous pouvons la déterminer par optimisation du poids du transformateur.\\
Cette fois-ci nous allons faire la résolution pour des sections de fenêtre allant de $0,1\times{}10^{-3} m²$ à $3\times{}10^{-3} m²$.\\

Voici l'évolution des masses obtenues : 
\begin{figure}[h]
	\begin{center}
	\includegraphics[width=0.5\textwidth]{images/TP1_plot_masses}
	\caption{Courbes des masses en fonction de la fenêtre de bobinage}\label{img:plotMasses}
	\end{center}
\end{figure}
\FloatBarrier 

Nous choisissons donc la section apportant la masse la plus faible. C'est une méthode de résolution itérative.\\
Après la résolution de l'équation de dimensionnement, nous obtenons les côtes suivantes : \\
\begin{itemize}
\item $S_{fer} = 0,0019 m² $
\item $b = 0,043 m $
\item $H = 0,043 m $
\item $hf = 0,0831 m $
\item $W_a = 1,8\times{}10^{-3} m² $
\end{itemize}~\\

On constate que la section de fenêtre des bobinages est légèrement inférieure au dimensionnement précédent tandis-que la section du noyau ferromagnétique a elle augmenté.\\

\newpage
Cela donne le dessin de transformateur suivant :\\
\begin{figure}[h]
	\begin{center}
	\includegraphics[width=0.5\textwidth]{images/TP1_transfo_poids}
	\caption{Dessin du transformateur dimensionné avec optimisation du poids}\label{img:dessinTransfoPoids}
	\end{center}
\end{figure}
\FloatBarrier 

Dans ce cas de figure nous avons les volumes et masses suivantes : \\
\begin{itemize}
\item $V_{fer} = 5,555 \times{}10^{-4} m^3$
\item $V_{cu} = 3,474 \times{}10^{-4} m^3$
\item $M_{fer} = 4,366 kg$
\item $M_{cuivre} = 3,099 kg $
\item $M_{totale} = 7,465 kg $
\end{itemize}~\\

La masse ainsi obtenue n'est que légèrement plus faible mais elle est optimale. Le volume de cuivre (matériau plus dense que le fer) a été diminué. Il reste à voir si notre transformateur le plus léger est bien conforme à notre cahier des charges.\\

\subsection{Dimensionnement des bobinages}
Pour déterminer le nombre de spires nécessaires, nous allons exprimer $n_2$ en fonction de $W_a$, $K_u$ et $s_2$ qui sont des variables connues.\\
Comme $ m = \cfrac{V_2}{V_1} = \cfrac{n_2}{n_1} = \cfrac{I_1}{I_2} = \cfrac{J.s_1}{J.s_2}$ \\
On a $ \cfrac{n_2}{n_1} = \cfrac{s_1}{s_2} $, soit $ n_2.s_2 = n_1.s_1 $\\
Partant de là, on peut en déduire que $n_1 = \cfrac{W_a.K_u}{s_1.2}$ et $n_2 = \cfrac{W_a.K_u}{s_2.2}$\\
Ce qui nous donne en pratique : $n_1 = 305$ et $n_2 = 552$. Et assez logiquement, le rapport $\cfrac{W_a}{W_p}=0.5$.\\

On peut alors déduire les résistances des enroulements, à partir de la section de cuivre et de la longueur moyenne du fil. On trouve des résistances de : $R_1 = 1.49\Omega$ et $R_2 = 0.6\Omega$.\\

\newpage
\subsection{Validation du modèle}
Maintenant que nous avons dimensionné le transformateur, nous allons vérifier son comportement magnétique vis à vis du cahier des charges.\\
Pour cela nous devons calculer le courant magnétisant, le courant à vide du transformateur (qui devrait conduire à une induction $B_{max}$ de 1T en simulation).\\

Après le dessin 2D de notre transformateur, optimisé pour être le plus léger possible et après configuration de tous les paramètres du modèles (matériaux, valeurs aux limites, etc), nous obtenons les résultats magnétiques suivant :\\

\begin{figure}[h]
	\begin{center}
	\includegraphics[width=0.7\textwidth]{images/TP1_FEMM_validation}
	\caption{Simulation par méthode des éléments finis}\label{img:FEMMvalidation}
	\end{center}
\end{figure}
\FloatBarrier 

Comme le montre bien le gradient de couleur, l'induction moyenne dans notre transformateur est de 1T, conformément au cahier des charges. On peut ainsi conclure que notre dimensionnement est correcte d'un point de vue magnétique.\\