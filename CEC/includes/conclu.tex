\section{Conclusion}
On aura vu au cours de ce TP quelles sont les étapes clés d'un dimensionnement.\\
Partir d'un cahier des charges et des équations régissant le système, on a construit le modèle et définit les équations de dimensionnement.\\
Selon les degrés de libertés que nous offrent ces équations, on aura du fixer des contraintes et/ou faire des choix d'optimisation.\\

Toutefois, on aura remarqué que même si un certain choix d'optimisation peut répondre au cahier des charges sous un certain aspect (ici l'aspect magnétique) il peut être totalement invalide dans un autre domaine. On a eu ici l'exemple du transformateur optimisé en poids, qui serait en fait rapidement endommagé en utilisation normale de part ces caractéristiques thermiques extrêmes.\\
Il est donc important de traiter le problème dans sa globalité, au risque sinon de détruire l'appareil dès sa première utilisation.\\

L'intérêt de la simulation est évident ici. Elle nous a permis de constater un défaut grave, rapidement et sans engager de biens matériels. C'est une économie sur le plan temporel comme financier.\\
De plus, la simulation par la méthode des éléments finis nous aura permis de visualiser les répartitions internes de la thermique et du magnétisme, chose difficilement accessible en réalité.\\
On soulèvera aussi la précision de cette méthode à côté d'une méthode de calcul globale qui néglige les irrégularités.\\

Pour modérer notre problème de température, il faut savoir qu'actuellement, pour faciliter le refroidissement des transformateurs, l’enroulement parcouru par l’intensité la plus élevée est placé à l’extérieur.\\
Il est aussi possible, dans certains cas, d'avoir recours à un refroidissement forcé, afin de garder le transformateur dans des conditions normales de fonctionnement.