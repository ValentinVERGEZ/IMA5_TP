% Retrouver la question qui était posée XXX

Pour étudier le phénomène, appliquons la loi de Faraday qui donne la f.é.m. aux bornes du circuit : 

$e = n= \cfrac{d\varphi(t)}{dt}$

Si on néglige la résistance du circuit, ainsi que les fuites de flux, la source v compensera à chaque instant cette f.é.m. : 
$s = e = V_{max}\sin(\omega.t) = n\cfrac{d\varphi(t)}{dt}$
%Attention problème avec le omega t

On peut calculer le flux créé par le circuit à l'interieur du noyau par intégration :
$\varphi(t) = \int {\cfrac{V_{max}}{n} \sin(\omega.t) dt} = \cfrac{V_{max}}{n\omega}\cos((\omega.t)) + \varphi_0$
%Attention problème avec le omega t
La constante d'intégration $\varphi_0$ peut correspondre à la présence d'un aimant epermanent, à l'existance d'un flux rémanent ou encore à la présnece d'un autre enroulement alimenté en courant continu. 

Dans notre cas nous considerons que $\phi_0 = 0$

$\varphi(t) = - \cfrac{V_{max}}{n\omega} \cos(\omega.t) = \cfrac{V_{max}}{n\omega} \sin(\omega.t - \cfrac{\Pi}{2}$
%Attention problème avec le omega t

Le déphasage de $- \cfrac{\Pi}{2}$ indique que le flux est en quadrature retard sur la tension $v =  V_{max}\sin(\omega.t)$

D'après la relation $n\varphi = Li$, on en déduit que le courant passant dans la bobine s'écrit : 

$i = \cfrac{V_{max}}{L\omega} \sin(\omega.t - \cfrac{\Pi}{2}$, on retrouve ka définition de l'impédance d'une bobine parfait : $Z = L\omega$
%Attention problème avec le omega t

Si la section est constante, le champ B dans le noyau vaut : 

$B(t) =  \cfrac{\varphi(t)}{S} = \cfrac{V_{max}}{n\omega S}\sin(\omega.t - \cfrac{\Pi}{2}$
%Attention problème avec le omega S

On voit que la valeur maximale de ce champ est reliée à la valuer efficace de la f.é.m. par la relation :

$B_{max} =  \cfrac{V_{max}}{n\omega S} = \cfrac{sqrt{2}V_{eff}}{n 2\Pi f S}$ que l'on peut écrire : 

$V_{eff} = 4.44 n f S B_{max} = 4.44 n f \phi_{max}$

On appelle cette relation formule de Boucherot. Cette expression montre que la valeur maximale du flux $\phi_{max}$ ne dépend que de la valuer efficace $V_{eff} $ de la tension d'alimentation (à fréquence constante) La tension d'alimentation impose le flux et l'enroulement appelle un courant en conséquence. 