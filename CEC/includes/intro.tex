\section{Introduction}


\subsection{Un transformateur c'est quoi ? }
	
	Un transformateur monophasé est un convertisseur « alternatif-alternatif » qui permet de modifier l'amplitude d'une tension alternative en maintenant sa fréquence et sa forme inchangée. Sans lui le transport d'énergie électrique sur de grandes distances serait impossible. En effet grâce aux transformateurs élévateurs de tension, on transporte des ''volts'' plutôt que des ''ampères'', limitant ainsi les pertes d'énergie à quelques pourcents. De plus, les transformateurs sont des machines électriques statiques. L'absence de mouvement est d'ailleurs à l'origine de leur excellent rendement. Les transformateurs couvrent toutes les gammes de puissances et de tensions, de quelques VA pour l'alimentation de circuits électroniques, à quelques centaines de MVA pour l'alimentation ou le couplage des réseaux électrique. 

\subsection{La constitution d'un transformateur}
	
	Un transformateur électrique se compose de tôles ferromagnétiques et de bobinages. Les tôles constituent le circuit magnétique du transformateurs. Dans le milieu industriel, elles sont généralement de forme "E-I". 
	Les enroulements sont ensuite bobinés autour du noyau du circuit magnétique. Dans notre cas, un transformateur monophasé est constitué de deux bobinages, un primaire et un secondaire. 

 	%%% Voir figure XXX
 	
	Le circuit magnétique permet de "connecter"  magnétiquement le circuit primaire au secondaire en canalisant les lignes de champs magnétiques produites par le primaire.

\begin{figure}[ht]
	\begin{center}
	\includegraphics[width=0.2\textwidth]{images/TP_intro_transfo}
	\caption{Exemple de Transformateur monophasé}\label{img:Transfomono}
	\end{center}
\end{figure}
\FloatBarrier 


\subsection{Objectifs}
Le but de ce TP est de réaliser le dimensionnement d'un transformateur monophasé selon un cahier des charges donné puis de critiquer sa validité.\\
Le dimensionnement sera optimisé pour obtenir le transformateur le plus léger possible. On validera ensuite le modèle sur FEMM, un logiciel de simulation 2D par éléments finis. Enfin, nous finirions par une étude thermique du transformateur, afin de confirmer ou non notre choix d'optimisation.\\