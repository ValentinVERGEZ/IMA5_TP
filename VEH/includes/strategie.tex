\section{Stratégie}
%Stratégie : 
%Rappel de l'intérêt
%Explication du raisonnement en "mode"
%Brève explication du freinage
%Explication détaillée du mode boost
\subsection{Différents modes}
Précédemment on a pu voir que nos variables de réglages étaient au nombre de trois pour seulement une variable objectif et zéro variable de contrainte.\\
$V_r - (V_o + V_c) = 2$ \\
Cela signifie que l'on a de la liberté dans notre commande. On va pouvoir agir sur deux nouveaux paramètres.\\

La stratégie consiste à répartir les flux d'énergie dans le système, dans un but donné. Ici notre stratégie aura pour but d'économiser l'énergie en jouant sur deux paramètres : la répartition de couple entre le moteur thermique et le moteur électrique (afin d'optimiser le rendement du moteur thermique) ainsi que la répartition entre freinage électrique et freinage mécanique (afin de maximiser la récupération d'énergie sans dépasser les valeurs limites des composants de la partie électrique).\\

Pour résumer, nous avons identifié plusieurs mode de fonctionnement du système.\\
\begin{itemize}
\item Accélération toute électrique
\item Accélération toute thermique
\item Mode boost - Accélération avec apport électrique et thermique
\item Freinage purement mécanique
\item Freinage purement électrique (récupération d'énergie)
\item Freinage combiné mécanique / électrique
\end{itemize}~\\

Le choix des différents modes est représenté par l'algorithme suivant :  \\
\begin{algorithm}[H]
 \KwData{\\
 $N_{gearbox}$: Vitesse de rotation du moteur (tr/min)\\
 $F_{trans}$: Force appliquée à la transmission du véhicule\\
 $SOC$: Pourcentage de charge de la batterie\\
 $S_{conv}$: Gain du hacheur\\
 $K_{gear}$: Rapport de la boîte de vitesse\\}
 \KwResult{Mode de fonctionnement}
  \eIf{$F_{trans} < 0$}{
  	Freinage\\
  	 \uIf{$SOC >= SOC_{max}$ (Batterie pleinement chargée)}{
	 • Freinage mécanique
	 }
	 \uElseIf{$abs(S_{conv}) > 1$}{
	 • Freinage combiné
	 }
	\Else{
	• Freinage électrique}
	}
	{
	Accélération\\
	 \uIf{$N_{gearbox} < 1000 \text{ OU } K_{gear} = 0$ (Vitesse trop faible pour le moteur thermique)}{
	 • Mode électrique
	 }
	 \uElseIf{$SOC <= SOC_{min}$}{
	  •Mode thermique}
	\Else{
	• Mode BOOST}
	}
 \caption{Détermination du mode de fonctionnement}\label{algo:modes}
\end{algorithm}

%function MODE = strategyChoice(N_gearbox, F_trans, SOC, S_conv,I_arm, I_dcm, I_batt)
% 
%%%
%SOC_min = 0.4;
%SOC_max = 1;
%I_arm_max = 1e9;
%Idcm_max = 1e9;
%I_batt_max = 1e9;
% 
%%% 
%if F_trans < 0
%    % Recuperation d'energie
%    if SOC >= SOC_max
%        % Full mechanical brake
%        MODE = 7;
%    elseif abs(S_conv) > 1 || I_arm > I_arm_max || I_dcm > Idcm_max || I_batt > I_batt_max
%        % Mechanical brake assistance
%        MODE = 6;
%    else
%        % Full electrical brake  
%        MODE = 5;
%    end
%else
%    % FULL ICE ou FULL DCM ou BOOST
%    if N_gearbox < 1000
%        % FULL DCM
%        MODE = 2;
%    else
%        if SOC <= SOC_min
%            % FULL ICE
%            MODE = 3;
%        else
%            % BOOST
%            MODE = 4;
%        end
%    end
%end
%end

\subsection{Le mode boost}
Parmi ces six modes, quatre consistent à envoyer une consigne binaire. On ne s'attardera pas dessus.\\
Le mode freinage combiné consiste simplement à utiliser la partie électrique jusqu'à sa limite et de compléter avec le freinage mécanique, on ne détaillera pas non plus ce mode.\\

En revanche le mode boost est lui beaucoup plus intéressant, puisqu'il consiste à optimiser le rendement du moteur thermique, ce qui n'est pas chose évidente.\\
Notre idée a été de nous baser sur les caractéristiques de consommation de carburant par kWh en fonction du couple et de la vitesse.\\
Notre SMC impose la vitesse au moteur thermique. On possède une caractéristique de la consommation fonction de la vitesse et du couple. Connaissant la vitesse du moteur thermique, on peut extraire les consommation correspondant à cette vitesse. Ces consommations sont encore fonction du couple.\\
On choisit la consommation la plus faible et on en déduit le couple correspondant.\\
Ce couple n'est autre que notre couple de référence pour le moteur thermique.
Le moteur électrique nous sert alors à compléter l'apport en puissance nécessaire.\\

On observe l'effet de la stratégie sur la consommation de carburant et sur l'utilisation de la batterie (Figure \ref{img:liter_optimisation}).\\ On remarque que la batterie est beaucoup sollicitée lorsque cela est possible.\\

\begin{figure}[ht]
\begin{center}
	\includegraphics[width=0.7\textwidth]{images/liter_optimisation}
	\caption{Consommation de carburant et utilisation de la batterie (avec stratégie)}\label{img:liter_optimisation}
\end{center}
\end{figure}
\FloatBarrier