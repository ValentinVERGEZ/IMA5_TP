\section{Introduction}

\subsection{Objectif}

	Ce TP consiste à modéliser et contrôler le couplage parallèle thermique-électrique d'un véhicule électrique hybriqde (VEH). Pour cela, nous allons utiliser la Représentation Énergétique Macroscopique (REM) pour modéliser le système. Puis nous metterons en place une Structure Maximale de Commande (SMC) afin de controler XXX.\\
	Nous réaliserons aussi une commande de stratégie. Celle-ci aura pour but de (stratégie finale adoptée XXX).

\subsection{Hypothèses et simplifications}

	Le systèmes étant relativement complexe, nous avons formulé plusieurs hypothéses et simplifications. On obtient ainsi un système moins compliqué et suffisament proche de la réalité pour notre étude. 

\begin{enumerate} 
\item Les interrupteurs sont considérés comme idéaux.
\item Les aimants permanants de la machine à courant continu ne staturent pas (XXX).
\item L'inertie des différents arbres sont négligés.
\item L'inertie des roues est négligée.
\item L'dérence entre la roue et le sol est considérés parfaite. 
\item L'embrayage fonctionne en tout ou rien (sans glissement).
\end{enumerate}

\subsection{Contraintes}

	Nous avons considéré l'embrayage comme un ''interrupteur'' qui transmet ou non le couple fournir par le moteur. Partant de cette simplifications la voiture ne peut donc pas démarer à partir du moteur thermique. Par conséquent, il faudra démarer sur le moteur électrique. Une fois que l'on aura atteint une vitesse suffisante (pour ne pas faire caler le moteur), nous pourrons l'utiliser.

\begin{figure}[b]
	\begin{center}
	\includegraphics[width=0.7\textwidth]{images/Systeme_VEH.png}
	\caption{Représentation schématique du système}\label{img:Schéma système VEH}
	\end{center}
\end{figure}
\FloatBarrier 

% XXX je voudrais placer cette image juste après le texte et je n'y arrive pas !!!!
		
		
%Véhicule hybride
%Intérêt de la simulation (coût pour développement de la SMC/Strat)
%Intérêt de la strat