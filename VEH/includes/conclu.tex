\section{Conclusion}

\subsection{Le point sur la modélisation, la commande et la stratégie}
% Si tu pouvais me faire un résumé de deux trois phrases sur SMC et la stratégie

\subsection{Les conditions de test demandées}

Lors du test de notre véhicule électrique hybride, nous avons utilisé les conditions qui sont actuellement en vigueur pour tester la consommation des voitures. Ces conditions décrite par une directive européenne définissent un cycle de conduite automobile. Et bien que les constructeurs sont dans leur droit, les test de consommation ne représente pas la réalité. En effet ce cycle de conduite a été créé en juillet 1973 et donc conçu pour des voitures de cette époques. Par exemple, les accélérations demandées aux véhicules actuels sont totalement dépassées. Et l'on se retrouve souvent avec une consommation de l'ordre de 20\% supplémentaire en conditions réelles que ce qui est annoncé par le constructeur. Cependant un nouveau cycle de conduite devrait être mis en place durant l'année prochaine. 


\subsection{Les améliorations que l'on peut apporter}

Comme son nom l'indique, la SMC est la structure maximale de commande, or cette structure comportent deux inconvénient lorsque l'on veut concevoir réellement le système. Tout d'abords le nombre de capteurs utilisés est important . Ce qui entraine un coût non négligeable. De plus on utilise dans la SMC des variables auxquelles nous n'avons pas accès sur un modèle physique (exemple XXX). 

Pour pallier à ces deux problèmes, on doit donc dans un premier temps trouver un moyen de contourner les variables inaccessibles. On cherche donc à les estimer à l'aide d'autres données du systèmes. Puis dans un second temps on va essayer de condenser au maximum la SMC afin d'utiliser le moins de capteurs possible (observateur...). Cette opération complexifie grandement la structure de commande (XXX). 





%annexe : procedure de simulation