\section{Représentatiob énergétique Macroscopique du véhicule électrique hybrique}

\subsection{Équations et REM du système}

Si l'on veut construire la REM d'un système, il faut dans un premier temps écrire les équations qui régissent chaques éléments du sytèmes. 

%The system can be decomposed inti subsystems in interactions using 4 basic EMR elements (Table 1) :
Une fois cette étape réalisée, on construite la REM du sytèmes. Pour cela on décompose le système en sous-systèmes (XXX besoin d'aide pour finir cette phrase correctement)...

\subsection{Chopper}

\begin{wrapfigure}{r}{10cm}
\includegraphics[width=0.3\textwidth]{images/Chopper.png}
\end{wrapfigure}

$\begin{cases}
	 U_{chop} &= S_{conv}.U_{batt}\\
	I_{arm} &= S_{conv}.I_{dcm}
\end{cases}$


\subsection{Machine à courant continu}

\begin{wrapfigure}{r}{10cm}
\includegraphics[width=0.3\textwidth]{images/MCC.png}
\end{wrapfigure}

$\begin{cases}
	 T_{dcm} &= K_{cfiexc}.I_{dcm}\\
	E &= K_{cfiexc}.W_{dcm}\\
	I_{dcm} &= \cfrac{1}{L}\int{U_{ch}-E-R.I_{dcm}}	
\end{cases}$


\subsection{Boîte de vitesse}

\begin{wrapfigure}{r}{10cm}
\includegraphics[width=0.3\textwidth]{images/Gearbox.png}
\end{wrapfigure}

$\begin{cases}
	 T_{gear} &= T_{clutch}.K_{gear_{ratio}}.\mu{}_{gear}^{p}\\
	W_{gearbox} &= W_{pul}.K_{gear_{ratio}}
\end{cases}$


\subsection{Poulie}

\begin{wrapfigure}{r}{10cm}
\includegraphics[width=0.3\textwidth]{images/Pulley.png}
\end{wrapfigure}

$\begin{cases}
	W_pul &= W_{gearbox} = \cfrac{1}{K_p}.W_{dcm} \\
	T_pul &= (T_{gear}+K_p.T_{dcm}).\mu{}_p^{p}
\end{cases}$


\subsection{Différentiel}

\begin{wrapfigure}{r}{10cm}
\includegraphics[width=0.3\textwidth]{images/Differential.png}
\end{wrapfigure}

$\begin{cases}
	T_{Wh_R} &= T_{Wh_L} = \cfrac{T_{diff}}{2} \\
	W_{Wh_R} &= W_{Wh_L} = 2.W_{diff}
\end{cases}$\\
~\\
$\begin{cases}
	 T_{diff} &= K_{diff}.T_{pul}.\mu{}_{diff}^{p}\\
	W_pul &= K_{diff}.W_{diff}
\end{cases}$

\subsection{Roues}

\begin{wrapfigure}{r}{10cm}
\includegraphics[width=0.3\textwidth]{images/Wheel.png}
\end{wrapfigure}

$\begin{cases}
	F_{Wh_n} &= \cfrac{T_{Wh_n}}{R_{Wh_n}}  \\
	W_{Wh_n} &= \cfrac{V_{Wh_n}}{R_{Wh_n}}
\end{cases}$

\subsection{Châssis}

\begin{wrapfigure}{r}{10cm}
\includegraphics[width=0.3\textwidth]{images/Chassis.png}
\end{wrapfigure}

$ R_t $ : Rayon de braquage\\
$ l_ev $ : Largeur véhicule électrique\\
~\\
$\begin{cases}
	V_{L_{Wh}} &= \cfrac{R_t+\cfrac{l_ev}{2}}{R_t}.V_{ev} \\
	V_{R_{Wh}} &= \cfrac{R_t-\cfrac{l_ev}{2}}{R_t}.V_{ev}\\
	F_{ev} &= F_{R_{Wh}} + F_{L_{Wh}}
\end{cases}$\\
~\\Nous supposons que nous allons toujours tout droit, $ R_t = 0 $.\\
$\begin{cases}
	V_{L_{Wh}} &= V_{ev} \\
	V_{R_{Wh}} &= V_{ev}\\
	F_{ev} &= F_{R_{Wh}} + F_{L_{Wh}}
\end{cases}$\\
~\\
$\begin{cases}
	M_{hev}\cfrac{dV_{hev}}{dt} &= F_{hev}-F_{hext} \\
	V_{hev} &= \cfrac{1}{M_{hev}}\int{(F_{hev}-F_{hext})dt}
\end{cases}$
 
%Assez scolaire dans le raisonnement, des équations impliquent des blocs, implique la REM
%On insiste sur l'inversion du rendement selon la puissance.