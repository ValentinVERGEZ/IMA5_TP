%\section{SMC}
%SMC : 
%Inversion très scolaire aussi, on a les équations on les inverse. Quand la causalité ne peut pas être respecté, on remplace par un correcteur.
%Raisonnement sur les correcteurs.
\section{SMC - Structure maximale de commande}
Après avoir modélisé le véhicule via sa REM, on peut s'intéresser au contrôle. Une méthode efficace est d'utiliser le contrôle par inversion du modèle.\\

La Structure Maximale de Commande est une représentation unique de la chaîne de commande. Elle est optimale mais est aussi parfois difficile à réaliser en pratique.\\

\subsection{Chaîne de réglage}
La première chose à faire est d'identifier les variables à régler, appelées variables objectifs.\\
Dans notre cas, nous nous intéressons uniquement au contrôle de la vitesse du véhicule. La variable objectif est alors $V_{hev}$.\\

Nos variables de réglages sont : 
\begin{itemize}
\item $S_{conv}$ (Entrée de réglage du hacheur)
\item $T_{ice}$ (Couple du moteur thermique)
\item $F_{brake}$ (Force de freinage mécanique)
\end{itemize}

Les autres paramètres sont soit fixes soit contrôlés par l'utilisateur directement (exemple de $K_{gear}$, le rapport de la boîte de vitesse).\\

Nous n'avons pas de variable de contrainte, le moteur électrique étant à aimant permanent. Cette information sera utile pour la partie suivante, la stratégie.\\

%% Chaîne de réglage

\subsection{Inversion}
A partir des équations utilisées pour la REM précédemment, nous pouvons aisément inverser les blocs de la REM.\\

%% Exemple inversion

Toutefois, deux blocs posent problèmes. La partie accumulation du châssis ainsi que la partie accumulation du moteur à courant continu. En effet, une fois inversés ils présentent des dérivées ce qui en pratique nous obligerait à connaître l'état futur du système. Cela étant impossible, on remplace ces éléments dérivatifs par des correcteurs (PI ou IP).\\

\subsection{Correction}
Pour déterminer exactement les paramètres de ces correcteurs, on va développer la fonction de transfert en boucle fermée du système avec correcteur, choisir le comportement du système corrigé puis identifier les paramètres de notre correcteur.\\
\subsubsection{•}