\documentclass[10pt,oneside]{article}

%---------------------------------%
%__Inclusions de packages__%
%_Packages Basiques_%
  \usepackage[utf8]{inputenc}
  \usepackage[T1]{fontenc}
  \usepackage[francais]{babel}
  \usepackage{textcomp}

  \usepackage{bigcenter}
  \usepackage{placeins}

  \usepackage{verbatim} %Environnement pour Ècrire du code \begin{verbatimtab}[10] pour 10 espaces par tabulation
  \usepackage{moreverb}

%Inclure Images
  \usepackage{graphicx} % Pour l'ajout d'images (commande \includegraphics[height=200, width=600]{chemin})

%Maths
  \usepackage{amsmath} % Pour fraction complexes
  \usepackage{amssymb} %math
  \usepackage{mathrsfs} %math
  \usepackage{siunitx} % Notation complexe : \num{}

%Pour les figures
  \usepackage{wrapfig}
  \usepackage{tikz}
  \usepackage{float}

%---------------------------------%
%__Quelques configurations__%
%_Annexes_%
  \usepackage{appendix}\renewcommand{\appendixtocname}{Annexes} 		\renewcommand{\appendixpagename}{Annexes} 

%_Présentation et marges_%
  \usepackage{layout}
  \usepackage[top=2cm, bottom=3cm, left=2cm, right=2cm]{geometry} %Marges
  \pagestyle{plain} %Plain : Seulement numero en bas de page, sinon mettre headings
  \marginparwidth = 0.mm

%_LstListings_%
  %Package
    \usepackage{listings} %Environnement pour écrire du code    
    \usepackage{color}

	\definecolor{mygreen}{rgb}{0,0.6,0}
	\definecolor{mygray}{rgb}{0.5,0.5,0.5}
	\definecolor{mymauve}{rgb}{0.58,0,0.82}
  
  %Config langage Matlab
	\lstset{ %
	  inputencoding=latin1,
	  backgroundcolor=\color{white},   % choose the background color; you must add \usepackage{color} or \usepackage{xcolor}
	  basicstyle=\footnotesize,        % the size of the fonts that are used for the code
	  breakatwhitespace=false,         % sets if automatic breaks should only happen at whitespace
	  breaklines=true,                 % sets automatic line breaking
	  captionpos=b,                    % sets the caption-position to bottom
	  commentstyle=\color{mygreen},    % comment style
	  deletekeywords={...},            % if you want to delete keywords from the given language
	  escapeinside={\%*}{*)},          % if you want to add LaTeX within your code
	  escapechar=|,
	  extendedchars=true,              % lets you use non-ASCII characters; for 8-bits encodings only, does not work with UTF-8
	  frame=single,                    % adds a frame around the code
	  keepspaces=true,                 % keeps spaces in text, useful for keeping indentation of code (possibly needs columns=flexible)
	  keywordstyle=\color{blue},       % keyword style
	  language=Matlab,                 % the language of the code
	  morekeywords={*,...},            % if you want to add more keywords to the set
	  numbers=left,                    % where to put the line-numbers; possible values are (none, left, right)
	  numbersep=5pt,                   % how far the line-numbers are from the code
	  numberstyle=\tiny\color{mygray}, % the style that is used for the line-numbers
	  rulecolor=\color{black},         % if not set, the frame-color may be changed on line-breaks within not-black text (e.g. comments (green here))
	  showspaces=false,                % show spaces everywhere adding particular underscores; it overrides 'showstringspaces'
	  showstringspaces=false,          % underline spaces within strings only
	  showtabs=false,                  % show tabs within strings adding particular underscores
	  stepnumber=1,                    % the step between two line-numbers. If it's 1, each line will be numbered
	  stringstyle=\color{mymauve},     % string literal style
	  tabsize=2,                       % sets default tabsize to 2 spaces
	  title=\lstname                   % show the filename of files included with \lstinputlisting; also try caption instead of title
	}
  
  %Config Langage C
%    \lstset{ %Definition des paramètres du code
%    language=C,
%    basicstyle=\footnotesize,
%    numbers=left,
%    numberstyle=\normalsize,
%    numbersep=7pt,
%    showstringspaces=false,  % underline spaces within strings
%    showtabs=false,
%    frame=single, %Cadre autour du texte
%    escapeinside={\%*}{*)},         % if you want to add a comment within your code
%    morekeywords={*,...}            % if you want to add more keywords to the set
%    }
    
%_Commandes rapides pour maths_%
%\newcommand{\xz}{\ensuremath{\overrightarrow{x_0}}}
%\newcommand{\xu}{\ensuremath{\overrightarrow{x_1}}}
%\newcommand{\xd}{\ensuremath{\overrightarrow{x_2}}}
%\newcommand{\xt}{\ensuremath{\overrightarrow{x_3}}}
%\newcommand{\yz}{\ensuremath{\overrightarrow{y_0}}}
%\newcommand{\yu}{\ensuremath{\overrightarrow{y_1}}}
%\newcommand{\yd}{\ensuremath{\overrightarrow{y_2}}}
%\newcommand{\yt}{\ensuremath{\overrightarrow{y_3}}}
%\newcommand{\zz}{\ensuremath{\overrightarrow{z_0}}}
%\newcommand{\zu}{\ensuremath{\overrightarrow{z_1}}}
%\newcommand{\zd}{\ensuremath{\overrightarrow{z_2}}}
%\newcommand{\zt}{\ensuremath{\overrightarrow{z_3}}}

%---------------------------------%
%__Informations pour le document__%
  \title{\textsc{\textbf{TP - Conception et Eco-Conception}} - IMA5\\~\\
  \Large{\textit{Analyse du dimensionnement et étude d'un transformateur monophasé}}
  }
  \date{}
  \author{
   Pierre \bsc{Appercé}\\ \and
   Valentin \bsc{Vergez}
   }

%---------------------------------%
%__Début du document__%
\begin {document}

%_Première page
	% Titre du document	
	\maketitle

	% Table des matières
	\renewcommand{\contentsname}{Table des matières}
	\tableofcontents % Table des matières
	\newpage
	
%_Contenu
\section{Introduction}

\subsection{Objectif}

	Ce TP consiste à modéliser et contrôler le couplage parallèle thermique-électrique d'un véhicule électrique hybriqde (VEH). Pour cela, nous allons utiliser la Représentation Énergétique Macroscopique (REM) pour modéliser le système. Puis nous metterons en place une Structure Maximale de Commande (SMC) afin de controler XXX.\\
	Nous réaliserons aussi une commande de stratégie. Celle-ci aura pour but de (stratégie finale adoptée XXX).

\subsection{Hypothèses et simplifications}

	Le systèmes étant relativement complexe, nous avons formulé plusieurs hypothéses et simplifications. On obtient ainsi un système moins compliqué et suffisament proche de la réalité pour notre étude. 

\begin{enumerate} 
\item Les interrupteurs sont considérés comme idéaux.
\item Les aimants permanants de la machine à courant continu ne staturent pas (XXX).
\item L'inertie des différents arbres sont négligés.
\item L'inertie des roues est négligée.
\item L'dérence entre la roue et le sol est considérés parfaite. 
\item L'embrayage fonctionne en tout ou rien (sans glissement).
\end{enumerate}


\section{Représentatiob énergétique Macroscopique du véhicule électrique hybrique}

\subsection{Équations et REM du système}

Si l'on veut construire la REM d'un système, il faut dans un premier temps écrire les équations qui régissent chaques éléments du sytèmes. 

%The system can be decomposed inti subsystems in interactions using 4 basic EMR elements (Table 1) :
Une fois cette étape réalisée, on construite la REM du sytèmes. Pour cela on décompose le système en sous-systèmes (XXX besoin d'aide pour finir cette phrase correctement)...

\subsection{Chopper}
\vspace{-10px}
\begin{figure}[ht]
\centering
\begin{minipage}{.5\textwidth}  
\centering
$\begin{cases}
	 U_{chop} &= S_{conv}.U_{batt}\\
	I_{arm} &= S_{conv}.I_{dcm}
\end{cases}$
\end{minipage}~
\begin{minipage}{.5\textwidth}
  \centering
\includegraphics[height=80px]{images/Chopper.png}
\end{minipage}
\end{figure}
\FloatBarrier
\vspace{-20px}

\subsection{Machine à courant continu}
\vspace{-10px}
\begin{figure}[ht]
\centering
\begin{minipage}{.5\textwidth}  
\centering
$\begin{cases}
	 T_{dcm} &= K_{cfiexc}.I_{dcm}\\
	E &= K_{cfiexc}.W_{dcm}\\
	I_{dcm} &= \cfrac{1}{L}\int{U_{ch}-E-R.I_{dcm}}	
\end{cases}$
\end{minipage}~
\begin{minipage}{.5\textwidth}
  \centering
\includegraphics[height=80px]{images/MCC.png}
\end{minipage}
\end{figure}
\FloatBarrier
\vspace{-20px}

\subsection{Boîte de vitesse}
\vspace{-10px}
\begin{figure}[ht]
\centering
\begin{minipage}{.5\textwidth}  
\centering
$\begin{cases}
	 T_{gear} &= T_{clutch}.K_{gear_{ratio}}.\mu{}_{gear}^{p}\\
	W_{gearbox} &= W_{pul}.K_{gear_{ratio}}
\end{cases}$
\end{minipage}~
\begin{minipage}{.5\textwidth}
  \centering
\includegraphics[height=80px]{images/Gearbox.png}
\end{minipage}
\end{figure}
\FloatBarrier
\vspace{-20px}

\newpage
\subsection{Poulie}
\vspace{-20px}
\begin{figure}[ht]
\centering
\begin{minipage}{.5\textwidth}  
\centering
$\begin{cases}
	W_pul &= W_{gearbox} = \cfrac{1}{K_p}.W_{dcm} \\
	T_pul &= (T_{gear}+K_p.T_{dcm}).\mu{}_p^{p}
\end{cases}$
\end{minipage}~
\begin{minipage}{.5\textwidth}
  \centering
\includegraphics[height=100px]{images/Pulley.png}
\end{minipage}
\end{figure}
\FloatBarrier
\vspace{-30px}

\subsection{Différentiel}
\vspace{-0px}
\begin{figure}[ht]
\centering
\begin{minipage}{.5\textwidth}  
\centering
$\begin{cases}
	T_{Wh_R} &= T_{Wh_L} = \cfrac{T_{diff}}{2} \\
	W_{Wh_R} &= W_{Wh_L} = 2.W_{diff}
\end{cases}$\\
~\\
$\begin{cases}
	 T_{diff} &= K_{diff}.T_{pul}.\mu{}_{diff}^{p}\\
	W_pul &= K_{diff}.W_{diff}
\end{cases}$
\end{minipage}~
\begin{minipage}{.5\textwidth}
  \centering
\includegraphics[height=80px]{images/Differential.png}
\end{minipage}
\end{figure}
\FloatBarrier
\vspace{-10px}

\subsection{Roues}
\vspace{-20px}
\begin{figure}[ht]
\centering
\begin{minipage}{.5\textwidth}  
\centering
$\begin{cases}
	F_{Wh_n} &= \cfrac{T_{Wh_n}}{R_{Wh_n}}  \\
	W_{Wh_n} &= \cfrac{V_{Wh_n}}{R_{Wh_n}}
\end{cases}$
\end{minipage}~
\begin{minipage}{.5\textwidth}
  \centering
\includegraphics[height=100px]{images/Wheel.png}
\end{minipage}
\end{figure}
\FloatBarrier
\vspace{-30px}

\newpage
\subsection{Châssis}
\vspace{-0px}
\begin{figure}[ht]
\centering
\begin{minipage}{.5\textwidth}  
\centering
$ R_t $ : Rayon de braquage\\
$ l_ev $ : Largeur véhicule électrique\\
~\\
$\begin{cases}
	V_{L_{Wh}} &= \cfrac{R_t+\cfrac{l_ev}{2}}{R_t}.V_{ev} \\
	V_{R_{Wh}} &= \cfrac{R_t-\cfrac{l_ev}{2}}{R_t}.V_{ev}\\
	F_{ev} &= F_{R_{Wh}} + F_{L_{Wh}}
\end{cases}$\\
~\\Nous supposons que nous allons toujours tout droit, $ R_t = 0 $.\\
$\begin{cases}
	V_{L_{Wh}} &= V_{ev} \\
	V_{R_{Wh}} &= V_{ev}\\
	F_{ev} &= F_{R_{Wh}} + F_{L_{Wh}}
\end{cases}$\\
~\\
$\begin{cases}
	M_{hev}\cfrac{dV_{hev}}{dt} &= F_{hev}-F_{hext} \\
	V_{hev} &= \cfrac{1}{M_{hev}}\int{(F_{hev}-F_{hext})dt}
\end{cases}$
\end{minipage}~
\begin{minipage}{.5\textwidth}
  \centering
\includegraphics[height=100px]{images/Chassis.png}
\end{minipage}
\end{figure}
\FloatBarrier
\vspace{-30px}
 
%Assez scolaire dans le raisonnement, des équations impliquent des blocs, implique la REM
%On insiste sur l'inversion du rendement selon la puissance.
%\section{SMC}
%SMC : 
%Inversion très scolaire aussi, on a les équations on les inverse. Quand la causalité ne peut pas être respecté, on remplace par un correcteur.
%Raisonnement sur les correcteurs.
\section{SMC - Structure maximale de commande}
Après avoir modélisé le véhicule via sa REM, on peut s'intéresser au contrôle. Une méthode efficace est d'utiliser le contrôle par inversion du modèle.\\

La Structure Maximale de Commande est une représentation unique de la chaîne de commande. Elle est optimale mais est aussi parfois difficile à réaliser en pratique.\\

\subsection{Chaîne de réglage}
La première chose à faire est d'identifier les variables à régler, appelées variables objectifs.\\
Dans notre cas, nous nous intéressons uniquement au contrôle de la vitesse du véhicule. La variable objectif est alors $V_{hev}$.\\

Nos variables de réglages sont : 
\begin{itemize}
\item $S_{conv}$ (Entrée de réglage du hacheur)
\item $T_{ice}$ (Couple du moteur thermique)
\item $F_{brake}$ (Force de freinage mécanique)
\end{itemize}

Les autres paramètres sont soit fixes soit contrôlés par l'utilisateur directement (exemple de $K_{gear}$, le rapport de la boîte de vitesse).\\

Nous n'avons pas de variable de contrainte, le moteur électrique étant à aimant permanent. Cette information sera utile pour la partie suivante, la stratégie.\\

%% Chaîne de réglage

\subsection{Inversion}
A partir des équations utilisées pour la REM précédemment, nous pouvons aisément inverser les blocs de la REM.\\

%% Exemple inversion

Toutefois, deux blocs posent problèmes. La partie accumulation du châssis ainsi que la partie accumulation du moteur à courant continu. En effet, une fois inversés ils présentent des dérivées ce qui en pratique nous obligerait à connaître l'état futur du système. Cela étant impossible, on remplace ces éléments dérivatifs par des correcteurs (PI ou IP).\\

\subsection{Correction}
Pour déterminer exactement les paramètres de ces correcteurs, on va développer la fonction de transfert en boucle fermée du système avec correcteur, choisir le comportement du système corrigé puis identifier les paramètres de notre correcteur.\\
\subsubsection{•}
\section{Stratégie}
%Stratégie : 
%Rappel de l'intérêt
%Explication du raisonnement en "mode"
%Brève explication du freinage
%Explication détaillée du mode boost
\subsection{Différents modes}
Précédemment on a pu voir que nos variables de réglages étaient au nombre de trois pour seulement une variable objectif et zéro variable de contrainte.\\
$V_r - (V_o + V_c) = 2$ \\
Cela signifie que l'on a de la liberté dans notre commande. On va pouvoir agir sur deux nouveaux paramètres.\\

La stratégie consiste à répartir les flux d'énergie dans le système, dans un but donné. Ici notre stratégie aura pour but d'économiser l'énergie en jouant sur deux paramètres : la répartition de couple entre le moteur thermique et le moteur électrique (afin d'optimiser le rendement du moteur thermique) ainsi que la répartition entre freinage électrique et freinage mécanique (afin de maximiser la récupération d'énergie sans dépasser les valeurs limites des composants de la partie électrique).\\

Pour résumer, nous avons identifié plusieurs mode de fonctionnement du système.\\
\begin{itemize}
\item Accélération toute électrique
\item Accélération toute thermique
\item Mode boost - Accélération avec apport électrique et thermique
\item Freinage purement mécanique
\item Freinage purement électrique (récupération d'énergie)
\item Freinage combiné mécanique / électrique
\end{itemize}~\\

Le choix des différents modes est représenté par l'algorithme suivant :  \\
\begin{algorithm}[H]
 \KwData{\\
 $N_{gearbox}$: Vitesse de rotation du moteur (tr/min)\\
 $F_{trans}$: Force appliquée à la transmission du véhicule\\
 $SOC$: Pourcentage de charge de la batterie\\
 $S_{conv}$: Gain du hacheur\\
 $K_{gear}$: Rapport de la boîte de vitesse\\}
 \KwResult{Mode de fonctionnement}
  \eIf{$F_{trans} < 0$}{
  	Freinage\\
  	 \uIf{$SOC >= SOC_{max}$ (Batterie pleinement chargée)}{
	 • Freinage mécanique
	 }
	 \uElseIf{$abs(S_{conv}) > 1$}{
	 • Freinage combiné
	 }
	\Else{
	• Freinage électrique}
	}
	{
	Accélération\\
	 \uIf{$N_{gearbox} < 1000 \text{ OU } K_{gear} = 0$ (Vitesse trop faible pour le moteur thermique)}{
	 • Mode électrique
	 }
	 \uElseIf{$SOC <= SOC_{min}$}{
	  •Mode thermique}
	\Else{
	• Mode BOOST}
	}
 \caption{Détermination du mode de fonctionnement}\label{algo:modes}
\end{algorithm}

%function MODE = strategyChoice(N_gearbox, F_trans, SOC, S_conv,I_arm, I_dcm, I_batt)
% 
%%%
%SOC_min = 0.4;
%SOC_max = 1;
%I_arm_max = 1e9;
%Idcm_max = 1e9;
%I_batt_max = 1e9;
% 
%%% 
%if F_trans < 0
%    % Recuperation d'energie
%    if SOC >= SOC_max
%        % Full mechanical brake
%        MODE = 7;
%    elseif abs(S_conv) > 1 || I_arm > I_arm_max || I_dcm > Idcm_max || I_batt > I_batt_max
%        % Mechanical brake assistance
%        MODE = 6;
%    else
%        % Full electrical brake  
%        MODE = 5;
%    end
%else
%    % FULL ICE ou FULL DCM ou BOOST
%    if N_gearbox < 1000
%        % FULL DCM
%        MODE = 2;
%    else
%        if SOC <= SOC_min
%            % FULL ICE
%            MODE = 3;
%        else
%            % BOOST
%            MODE = 4;
%        end
%    end
%end
%end

\subsection{Le mode boost}
Parmi ces six modes, quatre consistent à envoyer une consigne binaire. On ne s'attardera pas dessus.\\
Le mode freinage combiné consiste simplement à utiliser la partie électrique jusqu'à sa limite et de compléter avec le freinage mécanique, on ne détaillera pas non plus ce mode.\\

En revanche le mode boost est lui beaucoup plus intéressant, puisqu'il consiste à optimiser le rendement du moteur thermique, ce qui n'est pas chose évidente.\\
Notre idée a été de nous baser sur les caractéristiques de consommation de carburant par kWh en fonction du couple et de la vitesse.\\
Notre SMC impose la vitesse au moteur thermique. On possède une caractéristique de la consommation fonction de la vitesse et du couple. Connaissant la vitesse du moteur thermique, on peut extraire les consommation correspondant à cette vitesse. Ces consommations sont encore fonction du couple.\\
On choisit la consommation la plus faible et on en déduit le couple correspondant.\\
Ce couple n'est autre que notre couple de référence pour le moteur thermique.
Le moteur électrique nous sert alors à compléter l'apport en puissance nécessaire.\\

On observe l'effet de la stratégie sur la consommation de carburant et sur l'utilisation de la batterie (Figure \ref{img:liter_optimisation}).\\ On remarque que la batterie est beaucoup sollicitée lorsque cela est possible.\\

\begin{figure}[ht]
\begin{center}
	\includegraphics[width=0.7\textwidth]{images/liter_optimisation}
	\caption{Consommation de carburant et utilisation de la batterie (avec stratégie)}\label{img:liter_optimisation}
\end{center}
\end{figure}
\FloatBarrier
\section{Conclusion}



%annexe : procedure de simulation

%%  Exemple inclusion image
%	\begin{figure}[h]
%	\begin{center}
%	 	\includegraphics[width=\textwidth]{test.jpg}
%	\end{center}
%	\caption{Titre test}\label{img:test}
%	\end{figure}

%_Annexes
	~\newpage{}
	\appendix
	\appendixpage
	\addappheadtotoc

%%  Exemple inclusion code Matlab
% NB : Pour un label, écrire "| \label{matlab:test} |"
%\lstinputlisting[language=Matlab,inputencoding=latin1,escapechar=|]{Matlab.m}
%\ref{matlab:test}
%\pageref{matlab:test}

\end{document}