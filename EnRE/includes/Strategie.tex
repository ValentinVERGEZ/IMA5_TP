 
\section{La stratégie}

\subsection{La commande MPPT}
Un Maximum Power Point Tracking (dispositif de poursuite du point de fonctionnement de puissance maximale), est un principe permettant de faire fonctionner un générateur électrique, dans notre cas les panneaux solaires, à la puissance maximale.

\begin{figure}[ht]
	\begin{center}
	\includegraphics[width=0.7\textwidth]{images/MPPT.jpeg}
	\caption{Exemple de stratégie MPPT}\label{img:courbe MPPT}
	\end{center}
\end{figure}
\FloatBarrier 

La Figure \ref{img:courbe MPPT} représente une caractéristique de puissance maximale en fonction de la tension et du courant (trait plein), ainsi que la caractéristique courant-tension typique d'un panneau photovoltaïque. Cette caractéristique signifie que l'on peut se placer à différentes tensions et toujours obtenir un fort courant. Mais au delà d'une tension critique, le courant s’effondre et la puissance avec. Il existe un point de fonctionnement courant-tension qui apporte la puissance maximale. C'est le MPP\footnote{Maximum Power Point} de la Figure \ref{img:courbe MPPT}.\\
Il faut noter que cette caractéristique évolue selon l'irradiance et la température auxquelles fonctionne le panneau photovoltaïque. Par conséquent, le point de fonctionnement à puissance maximal va se déplacer tout au long de l'utilisation du PV. L'objectif de la stratégie sera de faire toujours fonctionner le PV à ce point de fonctionnement, en d'autres mots de "suivre son déplacement".\\

Comme on le voit, cette caractéristique de puissance a une forme en cloche. Ça a pour effet négatif de nous empêcher de savoir s'il faut augmenter ou réduire la tension pour augmenter la puissance.\\
Pour pallier ce problème, on utilise une stratégie dite de "Perturb \& Observ" qui, comme son nom l'indique, va perturber le système (par des variations de tension) et observer le résultat pour déterminer dans quelle zone de la caractéristique on se situe. Ainsi, on sait s'il faut augmenter ou réduire la tension pour optimiser la puissance.\\


\begin{figure}[ht]
	\begin{center}
	\includegraphics[width=0.5\textwidth]{images/Energie_MPPT.png}
	\caption{Exemple de stratégie MPPT}\label{img:courbe energie MPPT}
	\end{center}
\end{figure}
\FloatBarrier 

La figure (\ref{img:courbe energie MPPT}) représente l'énergie fournie par les panneaux photovoltaïques lors de la simulation. Lorsque l'on compare cette énergie à l'énergie fournie par les panneux sans la stratégie MPPT au bout de 30 secondes, on obtient un écart de 600 Joules, soit 10\% de plus lorsque l'on utilise la stratégie MPPT.

Ces 10\% supplémetaires sont dues à la variation de Uc (controlé par le bloc MPPT), dont on peut voir l'évolution sur la figure \ref{img:courbe Uc MPPT}.

\begin{figure}[ht]
	\begin{center}
	\includegraphics[width=0.5\textwidth]{images/Uc_MPPT.png}
	\caption{Exemple de stratégie MPPT}\label{img:courbe Uc MPPT}
	\end{center}
\end{figure}
\FloatBarrier 

