\section{Modélisation de la chaîne de conversion}

Si l'on veut construire la REM d'un système, il faut dans un premier temps écrire les équations qui régissent chaque élément du système. Une fois cette étape réalisée, on construit les éléments de base de la REM du système correspondants aux équations. Puis on les assemblent pour composer une représentation complète du système.

\subsection{Les relations temporelles et REM du système}

\subsection{Bus continu PV}
%XXX
\vspace{-10px}
\begin{figure}[ht]
\centering
\begin{minipage}{.5\textwidth}  
\centering
$\begin{cases}
	 U_{c} &= \cfrac{1}{C}\bigints{(I_{pv}-I_L-\cfrac{U_c}{R_c}).dt}+CI
\end{cases}$
\end{minipage}~
\begin{minipage}{.5\textwidth}
  \centering
\includegraphics[height=80px]{images/Bus.png}
\end{minipage}
\end{figure}
\FloatBarrier
\vspace{-20px}

\subsection{Filtre PV}
%XXX
\vspace{-10px}
\begin{figure}[ht]
\centering
\begin{minipage}{.5\textwidth}  
\centering
$\begin{cases}
	 I_{L} &= \cfrac{1}{L}\bigints{(U_{c}-R_L.I_{L}-U_{hp}).dt}+CI
\end{cases}$
\end{minipage}~
\begin{minipage}{.5\textwidth}
  \centering
\includegraphics[height=80px]{images/Filtre.png}
\end{minipage}
\end{figure}
\FloatBarrier
\vspace{-20px}

\subsection{Hacheur parallèle}
%XXX
\vspace{-10px}
\begin{figure}[ht]
\centering
\begin{minipage}{.5\textwidth}  
\centering
$\begin{cases}
	I_{hp} &= m_{hp}.I_{L}.\eta{}_{hp}\\
	U_{hp} &= m_{hp}.U_{batt}
\end{cases}$
\end{minipage}~
\begin{minipage}{.5\textwidth}
  \centering
\includegraphics[height=80px]{images/Chopperp.png}
\end{minipage}
\end{figure}
\FloatBarrier
\vspace{-20px}

\subsection{Batteries}
%XXX
\vspace{-10px}
\begin{figure}[ht]
\centering
\begin{minipage}{.5\textwidth}  
\centering
$\begin{cases}
	 U_{batt} &= \cfrac{1}{C_{eq}}\bigints{(I_{hp}-I_{hs}).dt}+CI
\end{cases}$
\end{minipage}~
\begin{minipage}{.5\textwidth}
  \centering
\includegraphics[height=80px]{images/Batteries.png}
\end{minipage}
\end{figure}
\FloatBarrier
\vspace{-20px}

\subsection{Hacheur série}
%XXX
\vspace{-10px}
\begin{figure}[ht]
\centering
\begin{minipage}{.5\textwidth}  
\centering
$\begin{cases}
	 U_{hs} &= m_{hs}.U_{batt}.\eta{}_{hs}\\
	I_{hs} &= m_{hs}.I_{mcc}
\end{cases}$
\end{minipage}~
\begin{minipage}{.5\textwidth}
  \centering
\includegraphics[height=80px]{images/Choppers.png}
\end{minipage}
\end{figure}
\FloatBarrier
\vspace{-20px}

\subsection{Machine à courant continu}
%XXX
\vspace{-10px}
\begin{figure}[ht]
\centering
\begin{minipage}{.5\textwidth}  
\centering
$\begin{cases}
	C_{mcc} &= K_{em}.I_{mcc}\\
	E &= K_{em}.W_{mcc}\\
	I_{mcc} &= \cfrac{1}{L}\bigints{(U_{hs}-E-R.I_{mcc}).dt}+CI
\end{cases}$
\end{minipage}~
\begin{minipage}{.5\textwidth}
  \centering
\includegraphics[height=80px]{images/MCC.png}
\end{minipage}
\end{figure}
\FloatBarrier
\vspace{-20px}

\subsection{Arbre moteur}
%XXX
\vspace{-10px}
\begin{figure}[ht]
\centering
\begin{minipage}{.5\textwidth}  
\centering
$\begin{cases}
	 W_{arbre} &= \cfrac{1}{J}\bigints{(C_{mcc}-C_{pertes}-C_{pompes}).dt}+CI
\end{cases}$
\end{minipage}~
\begin{minipage}{.5\textwidth}
  \centering
\includegraphics[height=80px]{images/arbre.png}
\end{minipage}
\end{figure}
\FloatBarrier
\vspace{-20px}


\subsection{La vérification du comportement}

Pour vérifier le comportement du système nous avons "découpé" le système en deux sous-systèmes. Le premier allant des panneaux solaires jusqu'à la batterie. Le second de la batterie jusqu'à la pompe. Dans cette seconde partie nous avions remplacer la batterie par un bloc constant égal à la tension maximale de la batterie.\\
Nous nous sommes bien assurés que la puissance en entrée était, au rendement prêt, la puissance de sortie. Et nous avons aussi vérifié la cohérence des grandeurs.\\
% Si tu te souviens de quelque chose genre vérifiacation des courants / tension ou autres - (VVE) : J'ai blablaté ce que je pouvais sur les vérifs de puissance

Lors du test de l'ensemble du système, on remarque qu'au démarrage, la tension au niveau de la batterie chute. En effet, c'est à ce moment que la motopompe démarre et que l'irradiance est faible (nous la simulons par une rampe). La motopompe demande plus d’énergie que peuvent fournir les panneaux lors des premières secondes de simulation.\\
Cependant au bout qu'une dizaine de seconde, on voit la tension remonter ce qui signifie que la puissance délivrée par les panneaux photovoltaïques est supérieure à celle utilisée par la pompe. Le rôle de la batterie est très important car elle permet l'utilisation de la pompe même si l'irradiance n'est pas assez forte (sous réserve que la batterie soit assez chargée). 

\begin{figure}[ht]
	\begin{center}
	\includegraphics[width=0.5\textwidth]{images/Niveau_batt_inv.png}
	\caption{Évolution du niveau de la batterie au cours d'une simulation}\label{img:NivBatt}
	\end{center}
\end{figure}
\FloatBarrier 


\subsection{Estimation du rendement}
La REM facilite le calcul de puissance, puisqu'il suffit de multiplier les entrées ou sorties d'un bloc pour avoir la puissance en entrée ou en  sortie de celui-ci. À l'aide de simulink, on mesure les puissances en sortie de la batterie et en entrée de la motopompe. On obtient alors un rendement de $\mathbf{\eta = 73 \%}$.

En théorie, le rendement peut se calculer en multipliant les rendements de chaque bloc. Seuls les blocs Hacheur, Inductance et MCC présente des pertes. Elles sont dues, au rendement propre du hacheur, aux pertes résistives de l'induit et au couple de pertes du moteur à courant continu (que nous avons représenté au niveau de la MCC).\\
Ainsi le couple peut se déterminer de la manière suivante : \\
$\eta = \eta{}_{hs} \times \eta{}_{ind} \times \eta{}_{mcc} = \eta{}_{hs} \times \cfrac{U_{hs}-R.I_{mcc}}{U_{hs}} \times \cfrac{C_{pompe}}{C_{pompe}+C_{pertes}}$.\\
Pour rappel, dans notre cas $ U_{hs} = m_{hs}.\eta{}_{hs}.U_{batt} = 0.5\times0.9\times46 $ et $ I_{mcc} = \frac{1}{K_{em}}.C_{mcc} = \frac{1}{K_{em}}.(C_{pompe}+C_pertes)$\\
Donc $\eta = \eta{}_{hs} \times \cfrac{U_{hs}-R.\cfrac{1}{K_{em}}.(C_{pompe}+C_{pertes})}{U_{hs}} \times \cfrac{C_{pompe}}{C_{pompe}+C_{pertes}}$\\

Ici, $C_{pompe}$ est notre seul inconnu. Nous utilisons donc le Couple pompe de la simulation, qui est de $1,5Nm$ en régime permanent.\\
Le calcul ci-dessus, nous amène un rendement de $\mathbf{\eta = 73,5314498 \%}$. Ce qui est parfaitement conforme au résultat de la simulation.\\