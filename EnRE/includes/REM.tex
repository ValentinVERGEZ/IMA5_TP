\section{Modélisation de la chaîne de conversion}

Si l'on veut construire la REM d'un système, il faut dans un premier temps écrire les équations qui régissent chaque élément du système. Une fois cette étape réalisée, on construit les éléments de base de la REM du système correspondants aux équations. Puis on les assemblent pour composer une représentation complète du système.

\subsection{Les relations temporelles et REM du système}

\subsection{Bus continu PV}
%XXX
\vspace{-10px}
\begin{figure}[ht]
\centering
\begin{minipage}{.5\textwidth}  
\centering
$\begin{cases}
	 U_{chop} &= S_{conv}.U_{batt}\\
	I_{arm} &= S_{conv}.I_{dcm}
\end{cases}$
\end{minipage}~
\begin{minipage}{.5\textwidth}
  \centering
\includegraphics[height=80px]{images/Bus.png}
\end{minipage}
\end{figure}
\FloatBarrier
\vspace{-20px}

\subsection{Filtre PV}
%XXX
\vspace{-10px}
\begin{figure}[ht]
\centering
\begin{minipage}{.5\textwidth}  
\centering
$\begin{cases}
	 U_{chop} &= S_{conv}.U_{batt}\\
	I_{arm} &= S_{conv}.I_{dcm}
\end{cases}$
\end{minipage}~
\begin{minipage}{.5\textwidth}
  \centering
\includegraphics[height=80px]{images/Filtre.png}
\end{minipage}
\end{figure}
\FloatBarrier
\vspace{-20px}

\subsection{Hacheur parallèle}
%XXX
\vspace{-10px}
\begin{figure}[ht]
\centering
\begin{minipage}{.5\textwidth}  
\centering
$\begin{cases}
	 U_{chop} &= S_{conv}.U_{batt}\\
	I_{arm} &= S_{conv}.I_{dcm}
\end{cases}$
\end{minipage}~
\begin{minipage}{.5\textwidth}
  \centering
\includegraphics[height=80px]{images/Chopperp.png}
\end{minipage}
\end{figure}
\FloatBarrier
\vspace{-20px}

\subsection{Batteries}
%XXX
\vspace{-10px}
\begin{figure}[ht]
\centering
\begin{minipage}{.5\textwidth}  
\centering
$\begin{cases}
	 U_{chop} &= S_{conv}.U_{batt}\\
	I_{arm} &= S_{conv}.I_{dcm}
\end{cases}$
\end{minipage}~
\begin{minipage}{.5\textwidth}
  \centering
\includegraphics[height=80px]{images/Batteries.png}
\end{minipage}
\end{figure}
\FloatBarrier
\vspace{-20px}

\subsection{Hacheur série}
%XXX
\vspace{-10px}
\begin{figure}[ht]
\centering
\begin{minipage}{.5\textwidth}  
\centering
$\begin{cases}
	 U_{chop} &= S_{conv}.U_{batt}\\
	I_{arm} &= S_{conv}.I_{dcm}
\end{cases}$
\end{minipage}~
\begin{minipage}{.5\textwidth}
  \centering
\includegraphics[height=80px]{images/Choppers.png}
\end{minipage}
\end{figure}
\FloatBarrier
\vspace{-20px}

\subsection{Machine à courant continu}
%XXX
\vspace{-10px}
\begin{figure}[ht]
\centering
\begin{minipage}{.5\textwidth}  
\centering
$\begin{cases}
	 T_{dcm} &= K_{cfiexc}.I_{dcm}\\
	E &= K_{cfiexc}.W_{dcm}\\
	I_{dcm} &= \cfrac{1}{L}\int{U_{ch}-E-R.I_{dcm}}	
\end{cases}$
\end{minipage}~
\begin{minipage}{.5\textwidth}
  \centering
\includegraphics[height=80px]{images/MCC.png}
\end{minipage}
\end{figure}
\FloatBarrier
\vspace{-20px}

\subsection{Arbre moteur}
%XXX
\vspace{-10px}
\begin{figure}[ht]
\centering
\begin{minipage}{.5\textwidth}  
\centering
$\begin{cases}
	 U_{chop} &= S_{conv}.U_{batt}\\
	I_{arm} &= S_{conv}.I_{dcm}
\end{cases}$
\end{minipage}~
\begin{minipage}{.5\textwidth}
  \centering
\includegraphics[height=80px]{images/arbre.png}
\end{minipage}
\end{figure}
\FloatBarrier
\vspace{-20px}


\subsection{La vérification du comportement}
%XXX méthode pour vérifier le comportement d'une partie du système
%XXX Même chose pour l'autre partie
%XXX Fonctionnement général




