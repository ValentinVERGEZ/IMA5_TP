\section{Conclusion}

\subsection{Le point sur la modélisation, la commande et la stratégie}

Dans l'ensemble, la conception de la REM et de la SMC se sont déroulées sans problèmes, ce processus étant relativement bien métrisé à ce jour. \\
Pour la stratégie nous avons eu l'occasion d'utiliser un bloc MPPT qui était déjà fait. Les changements que nous devions faire était minimes. On avons pu observer le fonctionnement de ce bloc au cours de simulations et ainsi voir les bénéfices pour la  puissance XXX
 

\subsection{Les améliorations que l'on peut apporter}

Comme son nom l'indique, la SMC est la structure maximale de commande, or cette structure comporte deux inconvénient lorsque l'on veut concevoir réellement le système. Tout d'abord le nombre de capteurs utilisés est important. Ce qui entraîne un coût non négligeable. De plus, on utilise dans la SMC des variables auxquelles nous n'avons pas toujours accès sur un système réel. Par exemple les forces extérieures que l'on mesure dans la simulation ne sont pas mesurables en réalité, ou très difficilement. \\

Pour pallier ces deux problèmes, on doit donc dans un premier temps, trouver un moyen de contourner les variables inaccessibles. On cherche donc à les estimer à l'aide d'autres données du systèmes. Puis dans un second temps, on va essayer de simplifier au maximum la SMC, sans trop perdre en résultat, afin d'utiliser le moins de capteurs possible (observateurs...). Cette opération complexifie grandement le travail sur la structure de commande. 