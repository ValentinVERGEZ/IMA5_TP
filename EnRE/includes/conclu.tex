\section{Conclusion}

\subsection{Le point sur la modélisation, la commande et la stratégie}

Dans l'ensemble, la conception de la REM et de la SMC se sont déroulées sans problèmes, ce processus étant relativement bien maîtrisé à ce jour. \\
Pour la stratégie, nous avons eu l'occasion d'utiliser un bloc MPPT qui était déjà fait. Seul quelques changements minimes étaient nécessaires pour utiliser ce bloc. 
 

\subsection{Les améliorations que l'on peut apporter}

Comme son nom l'indique, la SMC est la structure maximale de commande, or cette structure comporte deux inconvénients lorsque l'on veut concevoir réellement le système. Tout d'abord le nombre de capteurs utilisés est important. Ce qui entraîne un coût non négligeable. De plus, on utilise dans la SMC des variables auxquelles nous n'avons pas toujours accès sur un système réel. Par exemple, la force électromotrice que l'on mesure dans la simulation n'est pas mesurables en réalité. En effet il s'agit d'une valeur interne à l'induit de la machine à courant continu et est donc inccessible. \\

Pour pallier ces deux problèmes, on doit donc dans un premier temps, trouver un moyen de contourner les variables trop difficiles à mesurer. On cherche donc à les estimer à l'aide d'autres données du systèmes. Puis dans un second temps, on va essayer de simplifier au maximum la SMC, sans trop perdre en résultat, afin d'utiliser le moins de capteurs possible (observateurs...). Mais cette opération complexifie grandement le travail sur la structure de commande. Mais amene une chaine de commande réaliste appeler Structure Pratique de Commande (SPC). \\